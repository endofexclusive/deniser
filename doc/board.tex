%% Example data sheet
%% Feel free to modify and use this file for any purpose, under
%% either the LaTeX Project Public License or under public domain.

% Options here are passed to the article class.
% Most common options: 10pt, 11pt, 12pt
\documentclass[10pt]{datasheet}

% Input encoding and typographical rules for English language
\usepackage[utf8]{inputenc}
\usepackage[english]{babel}
\usepackage[english]{isodate}

% tikz is used to draw images in this example, but you can
% also use \includegraphics{}.
\usepackage{tikz}
\usepackage{pgfplots}
\usepackage{circuitikz}
\usetikzlibrary{calc}
\usepackage[abs]{overpic}
\usepackage{siunitx}
\usepackage{float}

\newcommand{\proj}{Deniser board}

\newcommand\Warning{%
 \makebox[1.4em][c]{%
 \makebox[0pt][c]{\raisebox{.1em}{\small!}}%
 \makebox[0pt][c]{\color{red}\Large$\bigtriangleup$}}}%


% These define global texts that are used in headers and titles.
\title{\proj{}}
\author{Some Author}
\date{April 2021}
\revision{Revision 1}
\companylogo{\Huge $\nabla$ Fancy Company}

\begin{document}
\maketitle

\section{Features}
\begin{multicols}{2}
\begin{itemize}
\item{MachXO3D FPGA}
\item{48-pin DIP connector}
\item{46 $\times$ \SI{5}{V} compatible I/O}
\end{itemize}
\columnbreak
\begin{itemize}
\item{On-board voltage regulation}
\item{Powered from \SI{5}{V} source}
\item{User connector with \SI{3.3}{V} I/O}
\end{itemize}

\end{multicols}

\section{General Description}
The \proj{} is a 48-pin DIP form factor board designed around a Lattice
Semiconductor MachXO3D FPGA.
It was designed for use in Amiga computers, while it can also function as a
general small-footprint FPGA breakout board.

\section{Hardware Overview}

\begin{figure}[H]
\begin{center}
  \begin{includegraphics}[width=0.85\textwidth]{callout.pdf}
  \end{includegraphics}
\end{center}
    \caption{\proj{} with callouts}

\begin{tabularx}{\textwidth}{  l X | l X }
    \thickhline
    \textbf{Callout} & \textbf{Description} & \textbf{Callout} & \textbf{Description} \\
    \hline
    1 & JTAG header & 4 & MachXO3D FPGA\\
    2 & User LED                & 7, 9 & DIP connector\\
    3, 5, 6 & Bus transceiver \SI{5}{V} $\leftrightarrow$ \SI{3.3}{V} & 8 & Voltage regulator\\
    \hline
    \thickhline
\end{tabularx}
\end{figure}

\section{Board Power}
The \proj{} is powered with \SI{4.75}{V} to \SI{5.25}{V} DC from the DIP
connector.
Input power is routed to a Texas Instruments LP5907
linear voltage regulator.
This voltage regulator is rated for 250mA and produces
the main \SI{3.3}{V} supply for the board.

A LED (\texttt{D1}) indicates that the board is receiving power on the \SI{5}{V} power rail.

\begin{table}[ht]
\caption{Voltage regulator circuit information}
\begin{tabularx}{\textwidth}{l   l   l   X}
    \thickhline
    \textbf{Supply} & \textbf{Circuits} & \textbf{Device} & \textbf{Max. Current} \\
    \hline
    \SI{3.3}{V} & All & \texttt{U5}: Texas Instruments LP507 & \SI{250}{mA} \\
    \thickhline
\end{tabularx}
\end{table}

\section{FPGA Configuration}
After power-on, the MachXO3D FPGA must be configured before it can perform any
function.
The FPGA is normally configured from the on-chip non-volatile configuration
memory.
The JTAG interface allows configuring the FPGA at any time aswell as
programming the on-chip non-volatile configuration memory.
JTAG signals are available on the user expansion header \texttt{J3}.

\section{FPGA variants}
The \proj{} is compatible with MachXO3D devices specified for \SI{3.3}{V}
supply voltage, such as the ZC and HC device variants.
These devices have an internal linear voltage regulator, which is powered by
the \proj{} \SI{3.3}{V} power rail.

MachXO3D HE devices can not be used with the \proj{}.

\section{Block diagram}
\begin{figure}[ht]
\begin{center}
  \begin{includegraphics}[width=0.40\textwidth]{hw-block.pdf}
  \end{includegraphics}
    \caption{Hardware block diagram}
\end{center}
\end{figure}

\clearpage
\section{DIP connector}
\begin{table}[H]
\begin{threeparttable}
\caption{DIP connector pinout}
\begin{tabularx}{\textwidth}{ l l l l l }
  \thickhline
  \textbf{Pin} & \textbf{Designation} & \textbf{Function} & \textbf{Direction} & \textbf{Board connection}\\
  \hline
  01-07   & D6-D0         & Data bus lines 6 to 0     & I/O & \texttt{U3} \\
  08      & M1H           & Mouse 1 horizontal        & I & \texttt{U4} \\
  09      & M0H           & Mouse 0 horizontal        & I & \texttt{U4} \\
  10-17   & RGA8-RGA1     & Register address bus 8-1  & I & \texttt{U4} \\
  18      & /BURST        & Color burst               & O & \texttt{U2} \\
  19      & VCC           & +\SI{5}{V}                & I & LDO input \\
  20-23   & R0-R3         & Video red bits 0-3        & O & \texttt{U2} \\
  24-27   & B0-B3         & Video blue bits 0-3       & O & \texttt{U2} \\
  28-31   & G0-G3         & Video green bits 0-3      & O & \texttt{U2} \\
  32      & /CSYNC        & Composite sync            & I & \texttt{U4} \\
  33      & /ZD           & Background indicator      & O & \texttt{U2} \\
  34      & CDAC\tnote{1} & CDAC clock                & I (ECS Denise) & \texttt{U4} \\
  35      & 7M            & \SI{7.15909}{MHz}         & I & \texttt{U4} \\
  36      & CCK           & Color clock               & I & \texttt{U4} \\
  37      & VSS           & Ground                    & I & Ground \\
  38      & M0V           & Mouse 0 vertical          & I & \texttt{U4} \\
  39      & M1V           & Mouse 1 vertical          & I & \texttt{U4} \\
  40-48   & D15-D7        & Data bus lines 15 to 7    & I/O & \texttt{U3} \\
  \thickhline
\end{tabularx}
\begin{tablenotes}
\item[1]{\textit{"Not connected"} in OCS Denise}
\end{tablenotes}
\end{threeparttable}
\end{table}

The transceivers \texttt{U2}, \texttt{U3} and \texttt{U4} are located between
the DIP connector \SI{5}{V} logic signals and the FPGA \SI{3.3}{V} I/O.

\begin{itemize}
\item{\texttt{U2} drives signals FPGA$\rightarrow$DIP.}
\item{\texttt{U4} drives signals FPGA$\leftarrow$DIP.}
\item{\texttt{U3} drives signals FPGA$\leftarrow$DIP when the signal
\texttt{drd\_rl\_to\_fpga} is high, and FPGA$\rightarrow$DIP when low.
The board signal \texttt{drd\_noe} is connected to the \texttt{U3} active low input
\texttt{nOE}.
\texttt{drd\_rl\_to\_fpga} and
\texttt{drd\_noe}
can be controlled by the FPGA.}
\end{itemize}

\clearpage
\section{User Expansion}
\label{user}

The header \texttt{J3} provides connections for JTAG and user I/O.

\begin{table}[H]
\caption{\texttt{J3} pinout}
\label{j3pin}
\begin{tabular}[c]{ l l l }
    \thickhline
    \textbf{Pin} & \textbf{Signal} & \textbf{MachXO3D pin}\\
    \hline
    \texttt{1}  & INITN & 55 \\
    \texttt{2}  & DONE  & 54 \\
    \texttt{3}  & TMS   & 70 \\
    \texttt{4}  & 3V3   & \textit{many} \\
    \texttt{5}  & GND   & PAD \\
    \texttt{6}  & TCK   & 71 \\
    \texttt{7}  & TDO   & 2 \\
    \texttt{8}  & TDI   & 1 \\
    \texttt{9}  & user0 & 4 \\
    \texttt{10} & user1 & 7 \\
    \thickhline
\end{tabular}
\end{table}

A convenient way to access the FPGA JTAG port is to connect with
the FTDI FT2232H Mini Module (Table \ref{mini}).

\begin{table}[H]
\caption{Connection reference for the FT2232H Mini Module}
\label{mini}
\begin{tabularx}{\textwidth}{ l r l X }
    \thickhline
    \textbf{Signal} & \textbf{\proj{} \texttt{J3}} & \textbf{FT2232H Mini Module} & \textbf{FT2232H} \\
    \hline
    TCK & 6   & CN2-7   & ADBUS0 \\
    TDI & 8   & CN2-10  & ADBUS1 \\
    TDO & 7   & CN2-9   & ADBUS2 \\
    TMS & 3   & CN2-12  & ADBUS3 \\
    \hline
    GND & 5   & CN2-2, CN2-4, CN2-6   & GND \\
    \thickhline
\end{tabularx}
\end{table}

\clearpage
\section{Electronic Schematics}
\label{sch}
%\begin{figure}[ht]
\begin{center}
  \begin{includegraphics}[width=1.0\textwidth]{deniser-schematic-v1.pdf}
%  \caption{Schematics}
  \end{includegraphics}
\end{center}
%\end{figure}

\clearpage
\section{Bill of materials}
\begin{table}[H]
\caption{Bill of materials}
\begin{tabularx}{\textwidth}{| l | l | X | l | X |}
    \thickhline
    \textbf{Item} & \textbf{\#} & \textbf{Designator} & \textbf{MPN} & \textbf{Description} \\
    \hline
    1   &  25 &
C1, C2, C3, C4, \newline
C5, C6, C7, C8, \newline
C9, C10, C11, C12, \newline
C13, C14, C15, C16, \newline
C17, C18, C21, C22, \newline
C23, C24, C25, C26, \newline
C27
                                    & GRM155R71C104KA88     & \SI{100}{nF} \\
    \hline
    2   &  4  & C19, C20, C28, C29  & LMK107BJ106MALTD      & \SI{10}{\mu F}        \\
    \hline
    3   &  5  & R1, R2, R3, R4, R5  & ERJ-3EKF4701          & \SI{4.7}{k\ohm}       \\
    \hline
    4   &  2  & D1, D2              & LTST-C191KRKT         & Red LED               \\
    \hline
    5   &  1  & U1                  & LCMXO3D-4300HC-5SG72C & FPGA                  \\
    \hline
    6   &  3  & U2, U3, U4          & 74LVCH162245ADGG      & 16-bit transceiver \\
    \hline
    7   &  1  & U5                  & LP5907MFX-3.3/NOPB    & LDO \SI{3.3}{V} \SI{250}{mA} \\
    \hline
    8   &  2  & J1, J2              & M20-9992445           & Header 1 $\times$ 24  \\
    \hline
    9   &  1  & J3                  & 10129381-910001BLF    & Header 2 $\times$ 5   \\
    \thickhline
\end{tabularx}
\end{table}

\end{document}

